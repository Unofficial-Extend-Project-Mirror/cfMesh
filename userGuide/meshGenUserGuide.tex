\documentclass[11pt,a4paper,dvips,oneside]{article}
\usepackage{setspace}
\usepackage{epsfig}
\usepackage{framed}
\usepackage[normalem]{ulem}
\usepackage{color}
\usepackage[a4paper,left=2cm,right=3cm,top=2cm,bottom=2cm]{geometry}
%
\renewcommand{\baselinestretch}{1.0}
%
\begin{document}
\begin{center}
{\bf MESH GENERATION USER GUIDE} \\
\vspace{0.2in}
\uline{{\large Author: \bf Franjo Jureti\'{c}}}
\end{center}
%
\vspace{0.2in}
%
\begin{flushleft}
Mesh generation is a set of tools for mesh generation which are then assembled
into a mesh generation workflow. At the moment the only supported mesher is a
cartesian mesher which generates the mesh from an octree templated refined to
the user-specified tolerance.

The process of mesh generation starts from a surface triangulation (usually an
.stl file) and it is it also needed to create a meshDict file which contains
information about required cell resolution at various parts of the
geometry. The meshDict file is contained in the ``system'' directory of your
working case. The entries of the meshDict file are:
\begin{itemize}
\item {\bf surfaceFile} is the name of the file containing surface triangulation
  of your geometry. In order to capture corners and
  edges of the surface it is necessary to divide the surface into different
  partition which can be done by using surfaceFeaturesEdges. It is possible to
  specify the angle tolerance for edge marking.
\item {\bf maxCellSize} is the maximum cell size anywhere in the domain.
\item {\bf boundaryCellSize} is the size which applies to all boundary cells.
\item {\bf patchCellSize} allows the user to specify finer cell size at desired
  patches.
\item {\bf decomposePolyhedraIntoTetsAndPyrs} is used in case when the user do not want any arbitrary polyhedra in the mesh. With the option switched on, the arbitrary polyhedra are decomposed into tetrahedra and pyramids. 
\item {\bf keepCellsIntersectingPatches} is intended to help in generating
  meshes in thin regions. This option means that octree boxes intersected by the
  given patches will also be used as mesh cells and this help the user to get a
  singly-connected mesh in thin regions.
\item {\bf keepCellsIntersectingBoundary} is similar to the previous option, but
  with a difference that all boundary-intersected octree boxes are used as mesh cells.
\item {\bf checkForGluedMesh} is an option which can remove cells which make the two disjoint domain parts connected.
\item {\bf objectRefinements} contains the list of objects which can be used to refine some interior mesh parts. The supported objects are lines, sppheres, cones and boxes.
\item {\bf subsetFileName} is the name of the file containing surface subsets which can then be used for localised-mesh refinement near the surface.
\item {\bf subsetCellSize} a list which is used to set the cell size for a subset. 
\end{itemize}

\end{flushleft}
%
%
\end{document}
